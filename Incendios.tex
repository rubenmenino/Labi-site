\documentclass[a4paper,11pt,onecloumn,oneside]{article}

\usepackage[utf8]{inputenc}
\usepackage[portuguese]{babel}
\usepackage[T1]{fontenc}
\usepackage{Incendios}

\setromanfont[Mapping=tex-text]{Linux Libertine O}
% \setsansfont[Mapping=tex-text]{DejaVu Sans}
% \setmonofont[Mapping=tex-text]{DejaVu Sans Mono}

\title{Incendios}
\author{Renato Lima Valente \\Ruben Miguel Paulo Menino}
\date{\today}

\begin{document}
\maketitle

\part{Abreviaturas de Siglas}

\begin{description}
 \item [G.N.R]- Guarda Nacional Republicana
 \item [ICNF] - Instituto de Conservação da Natureza e das Florestas
 \item [CO] - monóxido de carbono
 \item [NO3] - óxido nítrico
 \item [CO2] - dióxido de carbono
 \item [COV] - compostos orgânicos voláteis
 \item [SO2] - dióxido de enxofre
 \item [NO2] - dióxido de nitrogénio
 \item [O3] - ozono DGS- Direção Geral de Saúde
 \item [LNES] - Linha Nacional de Emergência Social
\end{description}


\part{Dados estatisticos relacionados nos incendios}

\part{Prevençao de Incendios}

\part{Riscos para a saude resultantes dos incendios}

\chapter{Riscos para a saudo associados ao fumo produzido pelos incendios}
\chapter{Problemas relacionados com o excesso de calor}
\chapter{Problemas relacionados com as falhas no abastecimento de agua}

\section{Protocolos de desinfeçao da agua - medidas a incrementar}

\subsection{Procura de agua destinada a consumo humano}
\subsection{Desinfeçao quimica}

\part{Apoios a poopulaçao apos incendios}

\part{Refleçao critica}

\part{Conclusao}

\part{Bibliografia}

Existem diversas entidades responsáveis pelo estudo estatístico dos incêndios, como o Instituto Nacional de Estatística, Proteção Civil Portugal, Instituto de conservação da natureza e das florestas e G.N.R.

\end{document}
