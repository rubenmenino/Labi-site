\documentclass[a4paper,11pt,onecloumn,oneside]{article}

\usepackage[utf8]{inputenc}
\usepackage[portuguese]{babel}
\usepackage[T1]{fontenc}
\usepackage{Incendios}

\setromanfont[Mapping=tex-text]{Linux Libertine O}
% \setsansfont[Mapping=tex-text]{DejaVu Sans}
% \setmonofont[Mapping=tex-text]{DejaVu Sans Mono}

\title{Incendios}
\author{Renato Lima Valente \\Ruben Miguel Paulo Menino}
\date{\today}

\begin{document}
\maketitle

\part{Abreviaturas de Siglas}

\begin{description}
 \item [G.N.R]- Guarda Nacional Republicana
 \item [ICNF] - Instituto de Conservação da Natureza e das Florestas
 \item [CO] - monóxido de carbono
 \item [NO3] - óxido nítrico
 \item [CO2] - dióxido de carbono
 \item [COV] - compostos orgânicos voláteis
 \item [SO2] - dióxido de enxofre
 \item [NO2] - dióxido de nitrogénio
 \item [O3] - ozono DGS- Direção Geral de Saúde
 \item [LNES] - Linha Nacional de Emergência Social
\end{description}

\part{Introdução}
Este trabalho foi realizado no âmbito da unidade curricular Laboratórios de Informática, esta que é lecionado pelo Professor Paulo Barraca. Foi solicitado à turma a elaboração de grupos para posteriormente se proceder à realização de um trabalho. O tema escolhido pelo nosso grupo foi “Incêndios”.
Na elaboração deste trabalho pretende-se que seja recolhido o máximo de informação credível e pertinente.
Sendo assim, podemos definir como principais objetivos: adquirir conhecimentos relativamente ao tema em questão; elaborar estratégias de modo a processar a informação para que seja facilmente percetível e compreensível ao leitor; Ter conhecimentos dos riscos para a saúde associados ao fumo produzido pelos incêndios, dos problemas relacionados com o excesso de calor, dos problemas relacionados com as falhas no abastecimento de água, apoios à população após incêndios e prevenção de incêndios.
A informação foi inserida neste documento de uma forma sequencial e lógica para que exista uma boa organização dos conteúdos, facilitando a compreensão dos mesmos.


\part{Dados estatísticos relacionados nos incêndios}
Existem diversas entidades responsáveis pelo estudo estatístico dos incêndios, como o Instituto Nacional de Estatística, Proteção Civil Portugal, Instituto de conservação da natureza e das florestas e G.N.R.

\part{Prevenção de Incendios}
A cada ano que decorre o número de incêndios em Portugal é cada vez mais frequente, é importante desta forma apostar recursos na prevenção dos mesmos. A intervenção do homem na prevenção é essencial de modo a evitar a origem dos incêndios bem como evitar o desenvolvimento dos mesmos. Segundo a Proteção Civil existem algumas situações em que é necessária uma maior prevenção por parte da população, de modo a diminuir a incidência dos incêndios:
\begin{enumerate}
 \item Queimadas
 \item Lançamento de foguetes
 \item Utilização de fósforos e cigarros
 \item Fogueiras
 \item Piqueniques
 \item Apicultura
 \item Linhas elétricas
 \item Vias férreas
 \item Rede viária
 \item Máquinas ou equipamento de motor ou combustão
 \item Chaminés
\end{enumerate}

A Proteção Civil salienta a importância dos cuidados a ter na realização das queimadas, tendo em conta diversos aspetos que devem ser respeitados como, temperatura do ar ambiente, humidade, vento, vigilância e água de modo a prevenir casos de emergência, na realização de queimadas.

\part{Riscos para a saúde resultantes dos incêndios}
Os incêndios florestais são considerados catástrofes ambientais. Estas catástrofes são responsáveis por prejuízos económicos e ambientais acrescidos, mas acima de tudo constituem um risco para todas as populações e para a saúde das mesmas.

\chapter{Riscos para a saúde associados ao fumo produzido pelos incêndios}

\chapter{Problemas relacionados com o excesso de calor}


\chapter{Problemas relacionados com as falhas no abastecimento de água}

\section{Protocolos de desinfeção da água - Medidas a incrementar}

\subsection{Fervura de água destinada a consumo humano}
Grande parte dos microrganismos patogénicos e prejudiciais para a saúde do Homem não conseguem sobreviver e proliferar em ambientes com altas temperaturas. Assim, a fervura da água contaminada é um método executado muito frequentemente e que é bastante eficiente na eliminação de bactérias, fungos, parasitas, entre outras. De seguida a água deverá ser armazenada num recipiente fechado e colocada num ambiente fresco.

\subsection{Desinfeção Química}
A desinfeção da água é executada através de produtos à base de cloro e iodo. Normalmente, é utilizada uma lixivia comercial à escolha do consumidor para uma desinfeção adequada, desde que não contenha corantes, nem qualquer tipo de detergentes. 

\begin{tabular}{|l||c||r|}
\hline
Concentraçaao de lixivia 
em hipoclorito de sodio(\%)  &  1 litro  &  2 litros  \\
	1\%		     & 	4 gotas	 &  8 gotas   \\
	2\%		     &	2 gotas	 &  4 gotas   \\
	4\%		     & 	1 gota	 &  2 gotas   \\
	8\%		     &     -     &  1 gota    \\
	10\%		     &     -     &  1 gota    \\
\end{tabular}
Tabela 1: Uso doméstico de lixivias comerciais de acordo com a concentração de hipoclorito sódio e da quantidade de água.

\part{Apoios a população apos incêndios}

\begin{enumerate}
 \item Ministério da Justiça
 O Ministério da Justiça em conjunto com o Ministério de Planeamento e Infraestruturas e com o Ministério das Finanças desenvolveu um programa que consiste em dispensar procedimentos e taxas e possibilita o cancelamento oficioso de matrículas e registos de veículos danificados pelos incêndios, sem necessidade de um pedido dos interessados e quaisquer deslocações às Conservatórias ou ao Instituto da Mobilidade e dos Transportes.
 Ainda com o envolvimento da Autoridade Tributária, vem assim evitar que os titulares falecidos ou os familiares recebam notificações para liquidação do Imposto Único de Circulação.
 
 \item Segurança Social 
 Atribuição de “subsídios de caráter eventual, de concessão única ou de manutenção, de apoio aos indivíduos e às famílias que se encontrem em situação de carência ou perda de rendimento e que necessitem de proceder a despesas necessárias à sua subsistência ou à aquisição de bens imediatos e inadiáveis”[ CITATION Rep17 \l 2070 ], nomeadamente:
 \begin{itemize}
 
  \item Despesas com rendas em situações de alojamento para habitação temporária; 
  \item Aquisição de bens e serviços de primeira necessidade nas áreas de alimentação, vestuário, habitação, saúde, educação e transportes; 
  \item Aquisição de instrumentos de trabalho; 
  \item Aquisição de ajudas técnicas/produtos de apoio; 
  \item Aquisição de outros bens e serviços ou realização de despesas considerados necessários após avaliação pelos serviços competentes da segurança social.
  
 \end{itemize}
 
 \item Ministério da saúde
 \begin{itemize}
  \item Fornecimento de apoio psicológico por equipas de Saúde Mental Comunitária (psiquiatras, psicólogos clínicos, enfermeiros e técnicos de Serviço Social); 
  \item Avaliação dos efeitos na saúde da população exposta aos incêndios por Equipas de Saúde Pública, que contam com médicos de saúde pública e técnicos de saúde Ambiental;
  \item Nos Cuidados de Saúde Primários, os recursos humanos são ajustados à procura, quer de consultas de agudos quer de consultas programadas.
  
 \end{itemize}


 \end{enumerate}

\part{Reflexão crítica}

\part{Conclusão}

\part{Bibliografia}

Existem diversas entidades responsáveis pelo estudo estatístico dos incêndios, como o Instituto Nacional de Estatística, Proteção Civil Portugal, Instituto de conservação da natureza e das florestas e G.N.R.

\end{document}
