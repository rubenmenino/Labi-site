\documentclass[a4paper,11pt,onecloumn,oneside]{article}

\usepackage[utf8]{inputenc}
\usepackage[portuguese]{babel}
\usepackage[T1]{fontenc}
\usepackage{Incendios}

\setromanfont[Mapping=tex-text]{Linux Libertine O}
% \setsansfont[Mapping=tex-text]{DejaVu Sans}
% \setmonofont[Mapping=tex-text]{DejaVu Sans Mono}

\title{Incendios}
\author{Renato Lima Valente \\Ruben Miguel Paulo Menino}
\date{\today}

\begin{document}
\maketitle

\part{Abreviaturas de Siglas}

\begin{description}
 \item [G.N.R]- Guarda Nacional Republicana
 \item [ICNF] - Instituto de Conservação da Natureza e das Florestas
 \item [CO] - monóxido de carbono
 \item [NO3] - óxido nítrico
 \item [CO2] - dióxido de carbono
 \item [COV] - compostos orgânicos voláteis
 \item [SO2] - dióxido de enxofre
 \item [NO2] - dióxido de nitrogénio
 \item [O3] - ozono DGS- Direção Geral de Saúde
 \item [LNES] - Linha Nacional de Emergência Social
\end{description}


\part{Dados estatisticos relacionados nos incendios}
Existem diversas entidades responsáveis pelo estudo estatístico dos incêndios, como o Instituto Nacional de Estatística, Proteção Civil Portugal, Instituto de conservação da natureza e das florestas e G.N.R.

\part{Prevençao de Incendios}
A cada ano que decorre o número de incêndios em Portugal é cada vez mais frequente, é importante desta forma apostar recursos na prevenção dos mesmos. A intervenção do homem na prevenção é essencial de modo a evitar a origem dos incêndios bem como evitar o desenvolvimento dos mesmos. Segundo a Proteção Civil existem algumas situações em que é necessária uma maior prevenção por parte da população, de modo a diminuir a incidência dos incêndios:
\begin{enumerate}
 \item Queimadas
 \item Lançamento de foguetes
 \item Utilização de fósforos e cigarros
 \item Fogueiras
 \item Piqueniques
 \item Apicultura
 \item Linhas elétricas
 \item Vias férreas
 \item Rede viária
 \item Máquinas ou equipamento de motor ou combustão
 \item Chaminés
\end{enumerate}

A Proteção Civil salienta a importância dos cuidados a ter na realização das queimadas, tendo em conta diversos aspetos que devem ser respeitados como, temperatura do ar ambiente, humidade, vento, vigilância e água de modo a prevenir casos de emergência, na realização de queimadas.

\part{Riscos para a saude resultantes dos incendios}
Os incêndios florestais são considerados catástrofes ambientais. Estas catástrofes são responsáveis por prejuízos económicos e ambientais acrescidos, mas acima de tudo constituem um risco para todas as populações e para a saúde das mesmas.

\chapter{Riscos para a saudo associados ao fumo produzido pelos incendios}

\chapter{Problemas relacionados com o excesso de calor}


\chapter{Problemas relacionados com as falhas no abastecimento de agua}

\section{Protocolos de desinfeçao da agua - Medidas a incrementar}

\subsection{Fervura de agua destinada a consumo humano}
Grande parte dos microrganismos patogénicos e prejudiciais para a saúde do Homem não conseguem sobreviver e proliferar em ambientes com altas temperaturas. Assim, a fervura da água contaminada é um método executado muito frequentemente e que é bastante eficiente na eliminação de bactérias, fungos, parasitas, entre outras. De seguida a água deverá ser armazenada num recipiente fechado e colocada num ambiente fresco.

\subsection{Desinfeçao Quimica}
A desinfeção da água é executada através de produtos à base de cloro e iodo. Normalmente, é utilizada uma lixivia comercial à escolha do consumidor para uma desinfeção adequada, desde que não contenha corantes, nem qualquer tipo de detergentes. 

\part{Apoios a poopulaçao apos incendios}

\part{Refleçao critica}

\part{Conclusao}

\part{Bibliografia}

Existem diversas entidades responsáveis pelo estudo estatístico dos incêndios, como o Instituto Nacional de Estatística, Proteção Civil Portugal, Instituto de conservação da natureza e das florestas e G.N.R.

\end{document}
